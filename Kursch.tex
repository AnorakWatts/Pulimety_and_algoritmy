% !TeX spellcheck = en_GB
% "Станет проще"

\documentclass[a4paper,12pt]{article} % тип документа
\usepackage[left=2cm,right=2cm,
top=2cm,bottom=2cm,bindingoffset=0cm]{geometry}
% report, book

%  Русский язык

\usepackage[T2A]{fontenc}			% кодировка
\usepackage[utf8]{inputenc}			% кодировка исходного текста
\usepackage[english,russian]{babel}	% локализация и переносы


% Математика
\usepackage{amsmath,amsfonts,amssymb,amsthm,mathtools} 
\usepackage{tikz}
\usepackage{wasysym}

\begin{document} % начало документа
\fontsize{14}{16pt}\selectfont
%\maketitle
\tableofcontents 
\newpage
\section{Теоретическая часть}
\textbf
1.	Источник и грамматика\\ \\
Пусть задан некоторый алфавит А. Также задан ориентированный граф G, некоторым ребрам которого приписаны буквы данного алфавита. Выделено множество начальных и конечных вершин. Такая конструкция называется источником. Источник порождает язык, состоящий из слов, порождаемых путями из начальных вершин в конечные. Всякий язык, задаваемый грамматикой, порождается некоторым источником, и всякий язык, порождаемый источником, задается некоторой грамматикой.\\  \\
\textbf
2.	Конечные автоматы\\ \\
Конечный автомат – U = {A, Q, V, g, f}, где
\\A – входной алфавит,
\\V – выходной алфавит,
\\Q – алфавит состояний,
\\g: Q×A → Q (функция переходов),
\\f: Q×A → V (функция выходов).
\\Обычно конечные автоматы представляют в виде ориентированного графа, где точки – это определенные состояния, а стрелки — направления переходов.\\ \\
\newpage
\section{Постановка задачи}
Требуется найти число слов длины n в алфавите a, b, c, d, которые переводят данный в условии автомат из состояния 0 в состояние 2 (дана таблица переходов автомата). Выписать
все слова длины $n = 3$ проверить соответствие количества полученных слов результату,полученному в формуле.
\\
\begin{center}
	\begin{tabular}{ | l | l | l | l |  }
		\hline
		& 0 & 1 & 2 \\ \hline
		a & 1 & 2 & 0 \\ \hline
		b & 1 & 0 & 0 \\ \hline
		c & 0 & 2 & 0 \\ \hline
		d & 2 & 2 & 2 \\ \hline
	\end{tabular}
	
	\begin{tikzpicture}[scale=0.2]
	\tikzstyle{every node}+=[inner sep=0pt]
	\draw [black] (35,-49.9) circle (3);
	\draw (35,-49.9) node {$S_0$};
	\draw [black] (9.9,-11) circle (3);
	\draw (9.9,-11) node {$S_1$};
	\draw [black] (69.4,-12.5) circle (3);
	\draw (69.4,-12.5) node {$S_2$};
	\draw [black] (32.177,-50.91) arc (-73.98191:-220.3543:23.469);
	\fill [black] (7.82,-13.16) -- (6.92,-13.44) -- (7.68,-14.09);
	\draw (5.36,-42.39) node [left] {$a\mbox{ }b$};
	\draw [black] (71.5,-14.64) arc (41.08819:-126.30302:25.408);
	\fill [black] (71.5,-14.64) -- (71.65,-15.57) -- (72.4,-14.91);
	\draw (71.59,-50) node [right] {$d$};
	\draw [black] (36.323,-52.58) arc (54:-234:2.25);
	\draw (35,-57.15) node [below] {$c$};
	\fill [black] (33.68,-52.58) -- (32.8,-52.93) -- (33.61,-53.52);
	\draw [black] (12.743,-11.955) arc (69.00097:-3.33717:35.251);
	\fill [black] (35.3,-46.92) -- (35.85,-46.15) -- (34.85,-46.09);
	\draw (30.35,-24.44) node [right] {$d$};
	\draw [black] (12.481,-9.471) arc (119.03667:58.07508:53.658);
	\fill [black] (66.9,-10.84) -- (66.48,-10) -- (65.96,-10.84);
	\draw (39.91,-2.18) node [above] {$a\mbox{ }c\mbox{ }d$};
	\draw [black] (70.971,-9.958) arc (176.00538:-111.99462:2.25);
	\draw (75.75,-7.63) node [right] {$d$};
	\fill [black] (72.37,-12.2) -- (73.14,-12.76) -- (73.21,-11.76);
	\draw [black] (36.087,-47.104) arc (157.45376:117.33141:65.925);
	\fill [black] (36.09,-47.1) -- (36.86,-46.56) -- (35.93,-46.17);
	\draw (47.92,-26.29) node [left] {$a\mbox{ }b\mbox{ }c$};
	\end{tikzpicture}
\end{center}
\newpage
\section{Основной способ нахождения}
			{Составим граматику по данным условиям:}\\
			$S_0,S_1,S_2$ - состояния\\ \\
			$ S_0\rightarrow aS_1\:|\:bS_1\:|\:cS_0\:|\:dS_2\:|\:d $\\ \\ 
			$ S_1\rightarrow aS_2\:|\:bS_0\:|\:cS_2\:|\:dS_2\:|\:a\:|\:c\:|\:d$\\ \\
			$ S_2\rightarrow aS_0\:|\:bS_0\:|\:cS_0\:|\:dS_2\:|\:d $\\ \\
			{Опишем языки:}\\
			$ L_0,L_1,L_2 $ - языки\\ \\
			$L_0 = aL_1\cup bL_1 \cup cL_0 \cup dL_2 \cup d $\\ \\
			$L_1 = aL_2\cup bL_0 \cup cL_2 \cup dL_2 \cup a \cup c \cup d $\\ \\
			$L_2 = aL_0\cup bL_0 \cup cL_0 \cup dL_2 \cup d $\\ \\
			Применим производящую функцию: \\
			$L_i(z)=h(a_{j_1})L_{K_1}(z)+h(a_{j_2})L_{k_2}(z)+\cdots+h(a_{j_l})L_{k_l}(z)+h(a_{m_1})+\cdots+h(a_{m_r})$\\ \\
			Здесь $h(w)$-одночлен, который ставится в соответствие слову w из языка L
			\\ \\
			В итоге имеем систему уравнений:
			\[ \left\{
			\begin{aligned}
			L_0 & = zL_1 + zL_1 + zL_0 + zL_2 + z \\
			L_1 & = zL_2 + zL_0 + zL_0 + zL_2 + z + z + z \\
			L_2 & = zL_0 + zL_0 + zL_0 + zL_2 + z
			\end{aligned} \right.
			\Rightarrow 
			\left\{ \begin{aligned}
			L_0 & = 2zL_1 + zL_0 + zL_2 + z\\
			L_1 & = 3zL_2 + zL_0 + 3z\\
			L_2 & = 3zL_0 + zL_2 + z
			\end{aligned}  \right.
			\: \: \: \:(1)\]\\  
			Выразим $L_0$ из системы:\\
			$L_2=\frac{3z}{1-z}L_0+\frac{z}{1-z} \: \: \: \: (2)$\\ \\
			$L_1=\frac{8z^2+z}{1-z}L_0+\frac{3z}{1-z} \: \: \: \: (3) $
			\\ \\
			Подставим (2) и (3) в (1.1):
			$ \: \: L_0=-\frac{6z^2+z}{16z^3+4z^2+2z-1}=\frac{6z^2+z}{4(1-4x)(z-\frac{\sqrt{3}i-1}{4})(z-\frac{-1-\sqrt{3}i}{4})} $\\
			$ L_0=\frac{1}{4}\left(
			\frac{A}{1-4z}+\frac{B}{z-\frac{\sqrt{3}i-1}{4}}+\frac{C}{z-\frac{-1-\sqrt{3}i}{4}}
			\right)\:\:= \frac{1}{4}\left(
			\frac{A}{1-4z}+\frac{B}{z-\frac{1}{2}e^{\frac{2\pi i}{3}}}+\frac{C}{z-\frac{1}{2}e^{\frac{-2\pi i}{3}}}
			\right)
			\:\:=\\ \\= \frac{1}{4}\left(
			\frac{A}{1-4z}+\frac{2Be^{\frac{-2\pi i}{3}}}{1-2ze^{\frac{-2\pi i}{3}}}+\frac{2Ce^{\frac{2\pi i}{3}}}{1-2ze^{\frac{2\pi i}{3}}}
			\right) $
			\newpage
			Метод нелпределенных коэффициентов:\\
			$ A\left(
			1-2ze^{\frac{-2\pi i}{3}}
			\right)
			\left(
			1-2ze^{\frac{2\pi i}{3}}
			\right)+2Be^{\frac{-2\pi i}{3}}(1-4z)\left(
			1-2ze^{\frac{2\pi i}{3}}
			\right)+\\ \\+2Ce^{\frac{2\pi i}{3}}(1-4z)\left(
			1-2ze^{\frac{-2\pi i}{3}}
			\right) \:\:=\:\: 6z^2+z $\\ \\ \\
			$ A\left(
			1-2ze^{\frac{2\pi i}{3}}-2ze^{\frac{-2\pi i}{3}}+4z^2
			\right)
			+2Be^{\frac{-2\pi i}{3}}\left(1-2ze^{\frac{2\pi i}{3}}-4z+8z^2e^{\frac{2 \pi i}{3}}\right)+\\+2Ce^{\frac{2\pi i}{3}}(1-2ze^{\frac{-2\pi i}{3}}-4z+8z^2e^{\frac{-2 \pi i}{3}}) \:\:=\:\: 6z^2+z $
			\\ \\ \\
			$ A\left(
			1-2ze^{\frac{2\pi i}{3}}-2ze^{\frac{-2\pi i}{3}}+4z^2
			\right)
			+2B\left(e^{\frac{-2\pi i}{3}}-2z-4ze^{\frac{-2 \pi i}{3}}+8z^2\right)+\\ \\+2C(e^{\frac{2\pi i}{3}}-2z-4ze^{\frac{2\pi i}{3}}+8z^2) \:\:=\:\: 6z^2+z $\\ \\
			Соберем все коэффиценты вместе \\ \\
			при $ z^2:\: 4A+16B+16C=6\:\Rightarrow\; 2A+8B+8C=3\\ \\ $при $ z: -2A(e^{\frac{2 \pi i}{3}}e^{\frac{-2 \pi i}{3}})+2B(-2-4e^{\frac{-2 \pi i}{3}})+2C(-2-4e^{\frac{2 \pi i}{3}})=1 \Rightarrow \\ \\ \Rightarrow -4A \cos \left( \frac{2 \pi}{3} \right)-4B-8B \left( \cos \left( \frac{2 \pi}{3} \right)-i \sin \left( \frac{2 \pi}{3} \right) \right)-4C-8C \left( \cos \left( \frac{2 \pi}{3} \right)+i \sin \left( \frac{2 \pi}{3} \right) \right)=\:=1 \Rightarrow -4A \left( - \frac{1}{2} \right) -4B-8B\left( -\frac{1}{2} -i \frac{\sqrt{3}}{2} \right)-4C-8C\left( -\frac{1}{2} +i \frac{\sqrt{3}}{2} \right)=1 \Rightarrow\\ \\ \Rightarrow 2A+4Bi \sqrt{3}-4Ci \sqrt{3}=1\\ \\$
			При const: $ A+2B \left(e^\frac{-2 \pi i}{3} \right)+2C\left(e^\frac{2 \pi i}{3} \right)=0 \Rightarrow A+2B \left( -\frac{1}{2} -i \frac{\sqrt{3}}{2} \right)+2C \left( -\frac{1}{2} +i \frac{\sqrt{3}}{2} \right)=\:=0 \Rightarrow A-B(1+i \sqrt{3})+C(-1+i\sqrt{3})=0\\$ Метод Крамера:\\ \\
			$ \Delta=\begin{pmatrix}
			2 & 8 & 8\\
			2 & 4i \sqrt{3} & -4i \sqrt{3}\\
			1 & -1-i\sqrt{3} & -1+i\sqrt{3}
			\end{pmatrix}=-112i \sqrt{3}$\\
			$ \Delta_A=\begin{pmatrix}
			3 & 8 & 8\\
			1 & 4i \sqrt{3} & -4i \sqrt{3}\\
			0 & -1-i\sqrt{3} & -1+i\sqrt{3}
			\end{pmatrix}=-40i \sqrt{3} \Rightarrow A= \frac{\Delta_A}{\Delta}=\frac{5}{14} $\\
			$\Delta_B=\begin{pmatrix}
			2 & 3 & 8\\
			2 & 1 & -4i \sqrt{3}\\
			1 & 0 & -1+i\sqrt{3}
			\end{pmatrix}=4(-1-4i \sqrt{3}) \Rightarrow B=\frac{1+4i\sqrt{3}}{28i \sqrt{3}}$\\
		$\Delta_C=\begin{pmatrix}
		2 & 8 & 3\\
		2 & 4i \sqrt{3} & 1\\
		1 & -1-i\sqrt{3} & 0
		\end{pmatrix}=4(1-4i \sqrt{3}) \Rightarrow C=\frac{-1+4i\sqrt{3}}{28i \sqrt{3}}$\\
		$ A=\frac{5}{14};\:\:B=\frac{1+4i\sqrt{3}}{28i \sqrt{3}};\:\:C=\frac{-1+4i\sqrt{3}}{28i \sqrt{3}} $
		\newpage
		Представим $ L_0 $ в виде:\\
		$ L_0=\frac{P(z)}{Q(z)}=\frac{A}{1-\alpha^{-1}z}+\frac{B}{1-\beta^{-1}z}+\frac{C}{1-\gamma^{-1}z}\\ $
		$\\ \sum_{k=0}^{\infty}(A(\alpha^{-1})^k)+(B(\beta^{-1})^k)+(C(\gamma^{-1})^k)z^k $,где $ \alpha,\beta,\gamma-$ корни $ Q(z)\\ \\ $ Данный ряд следует из следующего разложения Тейлора:\\ \\
		$ \frac{1}{1-kz}=1+kz+k^2z^2+\cdots+k^nz^n=\sum_{n=0}^{\infty}k^nz^n $\\ \\
		Получаем ряд:\\
		\boxed{\sum_{n=0}^{\infty}\frac{1}{4}(\frac{5}{14}4^n+\frac{4(1+4i\sqrt{3})4^n}{28i\sqrt{3}(\sqrt{3}i-1)^n(1-\sqrt{3}i)}+\frac{4(4i\sqrt{3}-1)4^n}{28i\sqrt{3}(\sqrt{3}i-1)^n(1+\sqrt{3}i)})z^n}$\:\:\:\:(*) \\ \\ \\$
		Основной ответ: Формула (*)-позволяет искать кол-во слов длинны n, переводящие данный автомат из состояния 0 в состояние 2.\\Это коэффицент при $ z^n\\ \\ $
		При $ n=3 $:\\
		$ \frac{1}{4}\left( \frac{5*64}{14}-\frac{4*64(1+4i\sqrt{3})}{28i \sqrt{3}(\sqrt{3}i-1)^4}-\frac{4*64(-1+4i\sqrt{3})}{28i \sqrt{3}(\sqrt{3}i+1)^4} \right)=20 $\\ \\
		Таким образом данное решение показывает, что существует 20 слов, переводящих автомат из условия из состояния 0 в состояние 2.\\ Проверим решение перебором 
		\newpage
		\section{Решение перебором}
		Путем перебора были найдены следующие слова длины 3, переводящие данный в условии конечный автомат из состояния 0 в состояние 2:\\ \\
		cad, cbd, cac, cbc, caa, cba, add, acd, aad, bdd, bcd, bad, ccd, add, bdd, ddd, cdd, dad, dbd, dcd\\ \\ Всего здесь 20 слов $ \Rightarrow $ Выведенная формула\\
		\boxed{\sum_{n=0}^{\infty}\frac{1}{4}(\frac{5}{14}4^n+\frac{4(1+4i\sqrt{3})4^n}{28i\sqrt{3}(\sqrt{3}i-1)^n(1-\sqrt{3}i)}+\frac{4(4i\sqrt{3}-1)4^n}{28i\sqrt{3}(\sqrt{3}i-1)^n(1+\sqrt{3}i)})z^n}\\ позволяет ВЕРНО вычислить сколько существует слов переводящих автомат из условия из состояния 0 в состояние 2.
		\newpage
		\section{Заключение}
		Таким образом, в ходе данной работы были представлены два способа нахождения количества слов длины n, переводящих данный в условии автомат из состояния 0 в состояние 2.
		Первый способ заключался в составлении производящей функции языка, заданного грамматикой из условия задачи. Была составлена производящая функция и система уравнений, из которой была получена формула для нахождения слов длины n. С помощью этой формулы было найдено число слов длины 3. Второй способ заключался в полном переборе слов длины 3, переводящих автомат из состояния 0 в состояние 2. Число этих слов совпало с числом слов, найденным по формуле. 
		\newpage
		\section{Список Литературы}
		1.	Белоусов А. И., Ткачев С. Б. Дискретная математика. — М.: МГТУ, 2006.\\ \\
		2.	Джон Хопкрофт, Раджив Мотвани, Джеффри Ульман. Дискретная математика. — 2-е изд. — Вильямс, 2002. (Алгоритмы и методы. Искусство программирования).\\ \\
		3.	Серебряков В. А., Галочкин М. П., Гончар Д. Р., Фуругян М. Теория и реализация языков программирования  — М., МЗ-Пресс, 2006 г., 2-е изд.\\ \\
		4.	Теория автоматов / Э. А. Якубайтис, В. О. Васюкевич, А. Ю. Гобземис, Н. Е. Зазнова, А. А. Курмит, А. А. Лоренц, А. Ф. Петренко, В. П. Чапенко // Теория вероятностей. Математическая статистика. Теоретическая кибернетика. — М.: ВИНИТИ, 1976.
		
		
\end{document} % конец документа